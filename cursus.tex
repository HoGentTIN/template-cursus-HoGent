% Cursus template
\documentclass[11pt,fleqn,a4paper]{book}

\input{structure}


%TODO: Aanpassen van de auteurd
\author{Dr. Jens Buysse, Bert {Van Vreckem}}
%TODO: aanpassen titel
\title{Cursus }
%TODO: aanpassen academiejaar
\date{Academiejaar 2018-2019}

\hypersetup{
  pdftitle={\thetitle},
  pdfauthor={\theauthor}
}

\begin{document}

\thetitlepage

%----------------------------------------------------------------------------------------
%	COPYRIGHT PAGE
%----------------------------------------------------------------------------------------

\newpage
~\vfill
\thispagestyle{empty}

%TODO:; Aanpassen copyright
\noindent Copyright \copyright\ 2015-2019 Jens Buysse\\ % Copyright notice

\noindent \textsc{www.hogent.be}\\ % URL

\noindent \textit{Gegenereerd op \today} % Printing/edition date

%----------------------------------------------------------------------------------------
%	TABLE OF CONTENTS
%----------------------------------------------------------------------------------------

\usechapterimagefalse

\tableofcontents % Print the table of contents itself

\cleardoublepage % Forces the first chapter to start on an odd page so it's on the right

\setlength{\parindent}{0pt}

\def\R{\mathbb{R}}

\include{dankwoord}

%Hier kan je de verschillende hoofdstukken toevoegen. Je maakt een file *.tex aan in dezelfde map als waar dit bronbestand in staat
% en voegt toe aan de hand van \include{*.tex}
\include{aan-de-slag}


\begin{appendices}

\include{notatie}

\clearpage
\addcontentsline{toc}{chapter}{\textcolor{maincolor}{\IfLanguageName{dutch}{Bibliografie}{Bibliography}}}
\printbibliography

\clearpage
\addcontentsline{toc}{chapter}{\textcolor{maincolor}{Index}}
\printindex
\end{appendices}
\end{document}
